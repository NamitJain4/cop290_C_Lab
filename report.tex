\documentclass{article}
\usepackage[utf8]{inputenc}
\usepackage{titlesec}
\usepackage{lipsum}
\usepackage{setspace}
\usepackage{graphicx}

\title{\LARGE COP290 C\_LAB}
\author{
    Ayush Ahuja - 2023CS50911\\
    Namit Jain - 2023CS10483\\
    Rishabh Narayanan - 2023CS10577
}
\date{}

\begin{document}
\maketitle

\section{Introduction}

\subsection{GitHub Repository}
\noindent{\texttt{https://github.com/NamitJain4/cop290\_C\_Lab}}

\subsection{Demo Video}
\noindent{\texttt{https://drive.google.com/file/d/1kHnV5IJ8C1TFrRvg1IgF5HjhEigGgPMQ/view?usp=drive_link}}


\section{Summary of Code Implementation}

\singlespacing % Set line spacing to single

\subsection{common.h}
\begin{itemize}
    \item Defines the Node structure to represent spreadsheet cells, including dependencies, operators, and error handling fields.
    \item Declares global variables for spreadsheet dimensions, execution status, and a lookup table for cell references.
    \item Provides function prototypes for printing the sheet and updating node values.
\end{itemize}

\subsection{main.c}
\begin{itemize}
    \item Implements command parsing and spreadsheet initialization.
    \item Handles user inputs, including arithmetic expressions, function calls (SUM, MAX, etc.), and scrolling commands.
    \item Manages memory allocation for cells and updates the execution status after processing commands.
    \item Parses and processes formulas, ensuring error detection for invalid syntax, out-of-bounds cells, and circular dependencies.
\end{itemize}

\subsection{operations\_implementation.c}
\begin{itemize}
    \item Implements core arithmetic operations (+, -, *, /) and spreadsheet functions like SUM, AVG, STDEV, MIN, MAX and SLEEP.
    \item Uses topological sorting (sort) to determine computation order and detect cycles.
    \item The updateNode function updates a cell’s value based on dependencies, recalculating only affected nodes.
    \item Handles division-by-zero errors and circular dependencies, marking affected nodes as erroneous.
\end{itemize}

\subsection{print\_sheet.c}
\begin{itemize}
    \item Implements the spreadsheet display functionality, printing a subset of cells based on user scrolling.
    \item Manages display control variables (print\_arr[]) to enable/disable output and adjust visible regions.
    \item Ensures proper formatting of cell values, including error indicators (ERR).
    \item Supports user commands for scrolling (w, a, s, d) and direct cell focus (scroll\_to \<CELL\>).
\end{itemize}

\section{Challenges Faced}

\subsection{Parser Implementation}
\begin{itemize}
    \item Parsing from left to right using case analysis revealed numerous edge cases initially overlooked.
    \item Key challenges included handling signs preceding constants, distinguishing between signed and unsigned integers, and correctly processing cell row numbers that do not start at zero.
\end{itemize}

\subsection{Dependency Graph Construction}
\begin{itemize}
    \item Implementing topological sorting with cycle detection posed significant difficulties.
    \item Additionally, dynamic memory management introduced challenges, particularly in debugging segmentation faults caused by inadvertent dereferencing of NULL pointers.
\end{itemize}

\subsection{Sheet Rendering}
\begin{itemize}
    \item Extending the existing implementation to support additional functionalities required careful modification.
    \item Challenges included efficiently handling multi-type content (integers and error values) and correctly accessing data stored in a row-major format.
\end{itemize}

\subsection{File Linking and Modularization}
\begin{itemize}
    \item Structuring the codebase effectively required careful consideration of header file contents, function calls across modules, and ensuring seamless data accessibility across different components.
\end{itemize}

\section{Important Edge Cases and Error Scenarios Covered by Test Suite}

\begin{itemize}
    \item \textbf{A2=2-A1}: Handling arithmetic expressions with RHS containing one cell and one constant.
    \item \textbf{A1=-1++2}: Handling explicit mention of the sign.
    \item \textbf{A1=2/0}: Handling division by zero (All dependent cell contents become ERR).
    \item \textbf{A1=B1; B1=A1}: Handling cyclical dependency (Error msg: cycle detected).
    \item \textbf{A1=hello}: Invalid command (Error msg: unrecognized cmd).
    \item \textbf{A0=3}: Invalid cell (Error msg: unrecognized cmd).
    \item \textbf{A1=MAX(D6:B1)}: Invalid range (Error msg: unrecognized cmd).
\end{itemize}

\newpage
\section{Program Structure and Design Decisions}
\begin{figure}[h]
    \centering
    \includegraphics[width=\textwidth]{program_structure_diagram.png}
    \caption{Overall Program Structure and Design Decisions}
    \label{fig:program_structure}
\end{figure}

\end{document}
